\section{SOLUCIÓN}

\begin{align*}
    G(s)=\displaystyle \frac{3(s+1)}{s(s^2+2s+6)}&&H(s)=0.5
\end{align*}
    
Escriba el programa en MATLAB y consigne la solución obtenida luego de la implementación
\subsection{Obtener $G(s)H(s)$ en su forma zpk}

\begin{figure}[H]
    \begin{framed}
        \centering
        \begin{minipage}[t]{.45\textwidth}
            \textbf{Código}
            \begin{minted}{Matlab}
G = tf(3, [1 0]) * tf([1 1], [1 2 6]);
H = 0.5;

% 1. Obtener G(s)H(s) en su forma zpk
disp("G(s)H(s) en su forma zpk");
gh = zpk(G*H)
            \end{minted}
        \end{minipage}
        \quad
        %
        \begin{minipage}[t]{.45\textwidth}
            \textbf{Resultado}
            
            \includegraphics[width=\linewidth]{Sis/Sis_zpk.png}
        \end{minipage}
    \end{framed}
\end{figure}
\newpage
\subsection{Asumiendo una ganancia variable k}

\subsubsection{Construir el LDR de manera gráfica}
\begin{figure}[H]
    \begin{framed}
        \centering
        \begin{minipage}[t]{.45\textwidth}
            \textbf{Código}
            \begin{minted}{Matlab}
% 2. Asumiendo una ganancia variable k
figure('Name','Mapa del LDR');
% 2.1. Construir el LDR de manera gráfica
rlocus(gh);
            \end{minted}
        \end{minipage}
        \quad
        %
        \begin{minipage}[t]{.45\textwidth}
            \textbf{Resultado}
            
            \includegraphics[width=\linewidth]{Sis/Sis_LDR.png}
        \end{minipage}
    \end{framed}
\end{figure}

\subsubsection{Tabular (20 puntos)}
\begin{figure}[H]
    \begin{framed}
        \centering
        \begin{minipage}[t]{.35\textwidth}
            \textbf{Código}
            \begin{minted}{Matlab}
% y tabular (20 puntos)
k = linspace(0, 1, 20);
[r, k] = rlocus(gh, k);
raices = r';
ganancias = k';
tabla = table(raices, ganancias)
            \end{minted}
        \end{minipage}
        \quad
        %
        \begin{minipage}[t]{.55\textwidth}
            \textbf{Resultado}
            
            \includegraphics[width=\linewidth]{Sis/Sis_tabla.png}
        \end{minipage}
    \end{framed}
\end{figure}

\subsubsection{En la gráfica debe mostrar el valor de k en cada punto de}
\begin{enumerate}
    \item Encuentro

    {\it No hay}
    
    \item Separación

    {\it No hay}
    \newpage
    
    \item Intersección con el eje imaginario.
    \begin{figure}[H]
        \centering
        \begin{framed}
            \includegraphics[width=0.7\linewidth]{Sis/Sis_LDR_InterEjeImg.png}
        \end{framed}
    \end{figure}
\end{enumerate}

\subsection{Analizar la estabilidad del sistema y describir el comportamiento temporal en función de k.}
Todas las raíces se encuentran en la región real negativa. Entonces, el sistema es estable para todo valor de k.
\begin{align*}
    &0<k && -a,-b\pm c &&A_1e^{-at} + A_2e^{-bt}\sin{(ct+\theta)}
\end{align*}

\subsection{Elija un valor de k en cada uno de los intervalos definidos y demuestre su análisis hallando las raíces del sistema.}
Para el sistema $G(s)H(s) = \displaystyle \frac{1.5k(s+1)}{s(s^2+2s+6)}$ elegimos un k para su único intervalo (0<k).

\begin{figure}[H]
    \begin{framed}
        \centering
        \begin{minipage}[t]{.45\textwidth}
            \textbf{Código}
            \begin{minted}{Matlab}
% 4. Elija un valor de k en cada uno de los
% intervalos definidos y demuestre su análisis
% hallando las raíces del sistema.
disp("Intervalo 1");
disp('Raices para un k igual a 0.05');
raiz = rlocus(gh, 0.05)
disp("Intervalo 2");
disp('Raices para un k igual a 1');
raiz = rlocus(gh, 1)
            \end{minted}
        \end{minipage}
        \quad
        %
        \begin{minipage}[t]{.45\textwidth}
            \textbf{Resultado}
            
            \includegraphics[width=\linewidth]{Sis/Sis_kFijo.png}
        \end{minipage}
    \end{framed}
\end{figure}
Como se observa en el resultado de Matlab:
\begin{itemize}
    \item Para el único intervalo $0<k$ se eligió un valor de prueba 1. Como raíces, obtuvimos 1 raíz real y 1 complejo conjugada, como era de esperarse:
    \begin{itemize}
        \item [$-a:$] $-2.2106$
        \item [$-b\pm jc:$] $-0.8947\pm 2.5145$
    \end{itemize}
\end{itemize}

\subsection{Código completo}
\begin{tcolorbox}
    \inputminted{Matlab}{Sis/sis.m}
\end{tcolorbox}
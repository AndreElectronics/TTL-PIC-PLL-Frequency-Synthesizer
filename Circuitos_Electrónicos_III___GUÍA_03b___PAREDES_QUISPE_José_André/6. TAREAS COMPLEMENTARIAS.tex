\section{TAREAS COMPLEMENTARIAS}

Para analizar los parámetros de los sintetizadores de receptores de radio comercial FM, de TV comercial y de aparatos celulares móviles, es importante comprender las características y requisitos específicos de cada tipo de receptor.

\subsection{Receptores de Radio Comercial FM}

\begin{itemize}
    \item \textbf{Rango de Frecuencia:} 
    \begin{itemize}
        \item 88 a 108 MHz: Este es el rango de frecuencia estándar para la radio FM comercial.
    \end{itemize}
    \item \textbf{Estabilidad de Frecuencia:} 
    \begin{itemize}
        \item ±50 ppm: La estabilidad de frecuencia es crucial para mantener la sintonización precisa en el rango de FM.
    \end{itemize}
    \item \textbf{Ruido de Fase:} 
    \begin{itemize}
        \item Bajo ruido de fase: Es esencial para evitar la distorsión y mantener la calidad del audio. Típicamente, se espera que el ruido de fase esté por debajo de -100 dBc/Hz a 10 kHz de offset.
    \end{itemize}
    \item \textbf{Resolución de Frecuencia:} 
    \begin{itemize}
        \item 100 kHz o menor: Permite una sintonización precisa dentro del canal de FM.
    \end{itemize}
    \item \textbf{Tiempo de Bloqueo (Lock Time):} 
    \begin{itemize}
        \item Menos de 100 ms: El tiempo que tarda el PLL en estabilizarse en la frecuencia deseada.
    \end{itemize}
    \item \textbf{Desviación de Frecuencia:} 
    \begin{itemize}
        \item ±75 kHz: La desviación de frecuencia máxima permitida en la modulación de FM.
    \end{itemize}
\end{itemize}

\subsection{Receptores de TV Comercial}

\begin{itemize}
    \item \textbf{Rango de Frecuencia:} 
    \begin{itemize}
        \item 54 a 806 MHz: Dependiendo de la región y el estándar (NTSC, PAL, SECAM), el rango puede variar.
    \end{itemize}
    \item \textbf{Estabilidad de Frecuencia:} 
    \begin{itemize}
        \item ±10 ppm o mejor: La estabilidad es crucial para evitar la deriva de frecuencia y mantener la sintonización precisa.
    \end{itemize}
    \item \textbf{Ruido de Fase:} 
    \begin{itemize}
        \item Bajo ruido de fase: Similar a los receptores de FM, el ruido de fase bajo es importante para evitar la interferencia y mantener la calidad de la señal de video.
    \end{itemize}
    \item \textbf{Resolución de Frecuencia:} 
    \begin{itemize}
        \item 1 MHz o menor: Para sintonizar los diferentes canales de TV.
    \end{itemize}
    \item \textbf{Tiempo de Bloqueo (Lock Time):} 
    \begin{itemize}
        \item Menos de 50 ms: El tiempo de bloqueo rápido es necesario para cambiar de canal rápidamente.
    \end{itemize}
    \item \textbf{Sensibilidad del Receptor:} 
    \begin{itemize}
        \item Menos de -85 dBm: La sensibilidad adecuada asegura una recepción clara incluso en condiciones de señal débil.
    \end{itemize}
\end{itemize}

\subsection{Aparatos Celulares Móviles}

\begin{itemize}
    \item \textbf{Rango de Frecuencia:} 
    \begin{itemize}
        \item 700 MHz a 2600 MHz: Dependiendo de la banda de operación (2G, 3G, 4G, 5G), los rangos de frecuencia pueden variar.
    \end{itemize}
    \item \textbf{Estabilidad de Frecuencia:} 
    \begin{itemize}
        \item ±0.1 ppm: Los teléfonos móviles requieren una estabilidad de frecuencia extremadamente alta debido a la necesidad de mantener la sincronización con la red celular.
    \end{itemize}
    \item \textbf{Ruido de Fase:} 
    \begin{itemize}
        \item Muy bajo ruido de fase: Es crítico para mantener la calidad de la señal y evitar la interferencia con otros canales y dispositivos. Valores típicos están por debajo de -120 dBc/Hz a 10 kHz de offset.
    \end{itemize}
    \item \textbf{Resolución de Frecuencia:} 
    \begin{itemize}
        \item 1 kHz o mejor: Permite una sintonización precisa y ajuste fino de la frecuencia portadora.
    \end{itemize}
    \item \textbf{Tiempo de Bloqueo (Lock Time):} 
    \begin{itemize}
        \item Menos de 1 ms: Los dispositivos móviles deben poder cambiar frecuencias rápidamente para manejar el traspaso de llamadas y datos entre celdas.
    \end{itemize}
    \item \textbf{Ancho de Banda del Canal:} 
    \begin{itemize}
        \item 1.4 MHz a 20 MHz: Dependiendo del estándar (LTE, 5G), el ancho de banda del canal puede variar significativamente.
    \end{itemize}
    \item \textbf{Sensibilidad del Receptor:} 
    \begin{itemize}
        \item -100 dBm o mejor: La alta sensibilidad es necesaria para recibir señales débiles y mejorar la calidad de la llamada y la transmisión de datos.
    \end{itemize}
\end{itemize}

\subsection{Consideraciones Comunes}

\begin{itemize}
    \item \textbf{Linealidad y Desviación de la Fase:} Para todos estos sistemas, es crucial que los sintetizadores mantengan una linealidad alta y una desviación de fase mínima para asegurar la calidad de la señal.
    \item \textbf{Consumo de Energía:} Especialmente para dispositivos móviles, el consumo de energía es una consideración importante para prolongar la vida útil de la batería.
    \item \textbf{Interferencias y Aislamiento:} La capacidad de aislar y minimizar las interferencias es esencial para todos los dispositivos que operan en un entorno electromagnético denso.
\end{itemize}

\section{CUESTIONARIO FINAL}

\begin{enumerate} [label={\arabic*)}]
    \item Presentar los resultados del procedimiento y comparar con los simulados.\\
    Se obtuvieron los resultados esperados del informe previo en la sección de procedimiento.
    \item Fundamentar la multiplicación de frecuencia utilizando contadores y cuáles son sus límites.\\
    La división de frecuencia con contadores funciona al tener que esperar una cantidad exponencial de 2 (para contadores binarios) de pulsos en la entrada para tener un cambio de estado en la salida. El número de contadores que se coloquen en cadena determinará la cantidad de pulsos a esperar. Si se desea algún valor que no sea de la escala exponencial de 2, se requiere un circuito lógico adicional que controle el reinicio del conteo.
    \item ¿Qué componentes podrían mejorar la magnitud de respuesta de división y/o multiplicación de frecuencia?\\
    El PLL CD4046 usando en la práctica funciona utilizando un voltaje TTL, lo que evita su optimización en la magnitud de respuesta, al ser esta ya muy estable para los voltajes TTL  lógicos (5V activo y 0 inactivo). Además, esta estabilidad es compartida por el comparador de fase, lo que permite que filtros sencillos, como el empleado en la práctica de un cero y un polo, sean suficientes. Incluso, el fabricante recomiendo el uso de filtros de orden 1, pues los de orden superior no tienen un gran impacto en el funcionamiento de los circuitos con el PLL CD4046. Sin embargo, una mejora que se puso en práctica fue el uso de un microcontrolador para la división de frecuencias programable de amplio rango. Con un mejor microcontrolador y un circuito más complejo (como uno con doble oscilador y doble divisor de frecuencias, programable y no programable) se podría tener una resolución mínima, un muy grande ancho de banda (rango de frecuencia disponibles) y una respuesta veloz y estable.
    \item Indicar en que aplicaciones prácticas de estos circuitos.\\
    Los circuitos sintetizadores con PLL tienen una amplia gama de aplicaciones prácticas en diversas áreas de la electrónica.
    \begin{enumerate}
        \item Telecomunicaciones:
        \begin{itemize}
            \item Generación de Frecuencias: En sistemas de telecomunicaciones, los PLL se utilizan para generar señales de referencia de alta frecuencia y para sintetizar frecuencias de portadora precisas.
            \item Modulación y Demodulación: Los PLL se emplean en moduladores y demoduladores de frecuencia, fase y amplitud para asegurar la coherencia de fase entre las señales transmitidas y recibidas.
        \end{itemize}
        \item Radiofrecuencia (RF):
        \begin{itemize}
            \item Sintonización de Radio: En receptores de radio, los PLL se usan para sintonizar frecuencias específicas de estaciones de radio con gran precisión.
            \item Osciladores Locales: Los PLL se utilizan en osciladores locales de receptores y transmisores de RF para estabilizar y ajustar frecuencias.
        \end{itemize}
        \item Relojes de Sistemas Digitales:
        \begin{itemize}
            \item Generación de Relojes: En sistemas digitales, los PLL generan señales de reloj de alta frecuencia a partir de una referencia de baja frecuencia, asegurando que todos los componentes del sistema funcionen en sincronía.
            \item Distribución de Relojes: Los PLL también se utilizan para distribuir señales de reloj a diferentes partes de un sistema digital con mínimas variaciones de fase (jitter).
        \end{itemize}
        \item Sistemas de Audio y Video:
        \begin{itemize}
            \item Sincronización de Video: En sistemas de video, los PLL se usan para sincronizar la señal de video con la fuente de referencia, asegurando la estabilidad de la imagen.
            \item Sistemas de Audio Digital: En equipos de audio digital, los PLL ayudan a mantener la sincronización entre diferentes componentes, evitando distorsiones y pérdidas de calidad.
        \end{itemize}
        \item Redes de Comunicaciones:
        \begin{itemize}
            \item En redes de comunicación, los PLL se emplean para mantener la sincronización entre nodos de red, esencial para la transmisión de datos sin errores.
            \item Multiplexación por División de Frecuencia (FDM): Los PLL se utilizan para sintetizar las diferentes frecuencias portadoras en sistemas FDM, permitiendo la transmisión de múltiples señales a través de un solo canal.
        \end{itemize}
    \end{enumerate}:
\end{enumerate}